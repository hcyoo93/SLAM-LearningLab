\documentclass{IEEEtran}
\usepackage[utf8]{inputenc}
\usepackage{cite}
\usepackage{amsmath}
\usepackage{amsfonts}
\usepackage{amssymb}
\usepackage{float}
\newcommand\norm[1]{\left\lVert#1\right\rVert}
\usepackage[flushleft]{threeparttable}
\usepackage[urlcolor=black,linkcolor=black,citecolor=black,colorlinks=true]{hyperref}
\usepackage{graphicx}
\newcommand\bmat[1]{\begin{bmatrix}#1\end{bmatrix}}
\newcommand{\mat}[1]{\boldsymbol{#1}}

\title{Left-Invariant EKF for Robot Localization:\\%
A Mobile Robotics Project}
\author{}
%
\begin{document}
\maketitle

\input{sections/abstract}

\section{Introduction}
\input{sections/introduction}

%\section{Theoretical Background}
%\input{sections/theoretical_background}
% brief introduction of InEKF and theorems

\section{bias-corrected Left-Invariant EKF}
\input{sections/left_invariant_ekf}

\section{Simulation Results} \label{sec:simulation_results}
\input{sections/simulation_results}

\section{C++ Implementation}
\input{sections/cpp_impl}

\section{Conclusion}
\input{sections/conclusion}

%
\bibliographystyle{unsrt}
\bibliography{references}
%

\begin{appendices}\label{appendices}

\section{Derivation of Discrete Dynamics} \label{sec:dyn_sec}
\input{sections/dyn_derivation}

\section{Simulated Data} \label{sec:fake_data}
\input{sections/fake_data}

\end{appendices}

\end{document}

